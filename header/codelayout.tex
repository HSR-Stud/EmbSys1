%%%%%%%%%%%%%%%
% Code Layout %
%https://en.wikibooks.org/wiki/LaTeX/Source_Code_Listings
%%%%%%%%%%%%%%%

\definecolor{mygreen}{rgb}{0,0.6,0}
\definecolor{mygray}{rgb}{0.5,0.5,0.5}
\definecolor{mymauve}{rgb}{0.58,0,0.82}

\lstset{ %
    firstnumber=1,
    backgroundcolor=\color{white},   % choose the background color; you must add        \usepackage{color} or \usepackage{xcolor}
    basicstyle=\footnotesize,        % the size of the fonts that are used for the code
    breakatwhitespace=false,         % sets if automatic breaks should only happen at whitespace
    breaklines=true,                 % sets automatic line breaking
    captionpos=b,                    % sets the caption-position to bottom
    commentstyle=\color{mygreen},    % comment style
    deletekeywords={...},            % if you want to delete keywords from the given language
    otherkeywords={...},             % if you want to add more keywords to the set
    escapeinside={\%*}{*)},          % if you want to add LaTeX within your code
    extendedchars=true,              % lets you use non-ASCII characters; for 8-bits encodings only, does not work with UTF-8
    frame=single,	                 % adds a frame around the code
    keepspaces=true,                 % keeps spaces in text, useful for keeping indentation of code (possibly needs columns=flexible)
    keywordstyle=\color{blue},       % keyword style
    language=C++,                    % the language of the code   
    numbers=left,                    % where to put the line-numbers; possible values are (none, left, right)
    numbersep=5pt,                   % how far the line-numbers are from the code
    numberstyle=\tiny\color{mygray}, % the style that is used for the line-numbers
    rulecolor=\color{black},         % if not set, the frame-color may be changed on line-breaks within not-black text (e.g. comments (green here))
    showspaces=false,                % show spaces everywhere adding particular underscores; it overrides 'showstringspaces'
    showstringspaces=false,          % underline spaces within strings only
    showtabs=false,                  % show tabs within strings adding particular underscores
    stepnumber=2,                    % the step between two line-numbers. If it's 1, each line will be numbered
    stringstyle=\color{mymauve},     % string literal style
    tabsize=2,	                     % sets default tabsize to 2 spaces
    %title=\lstname                   % show the filename of files included with         \lstinputlisting; also try caption instead of title
}

\lstdefinestyle{customc++}{
    belowcaptionskip=1\baselineskip,
    %frame=L,
    xleftmargin=\parindent,
    language=C++,
    basicstyle=\footnotesize\ttfamily,
    keywordstyle=\bfseries\color{blue},
    commentstyle=\itshape\color{mygreen},
    identifierstyle=\color{black},
    stringstyle=\color{gray},
}

\lstdefinestyle{cppunit}{
    belowcaptionskip=1\baselineskip,
    %frame=L,
    xleftmargin=\parindent,
    language=C++,
    basicstyle=\footnotesize\ttfamily,
    keywordstyle=\bfseries\color{blue},
    keywordstyle=[2]\bf\color{black}, %not sure why \bf works, but it does
    commentstyle=\itshape\color{mygreen},
    identifierstyle=\color{black},
    stringstyle=\color{gray},
    keywords=[2]{  %Cpp Unit Keywords
        CPPUNIT_ASSERT,
        CPPUNIT_TEST,
        CPPUNIT_TEST_EXCEPTION,
        CPPUNIT_TEST_END,
        CPPUNIT_TEST_SUITE,
        CPPUNIT_TEST_SUITE_REGISTRATION,
        CPPUNIT_TEST_SUITE_END},
}

\lstdefinestyle{c++qt}{
    belowcaptionskip=1\baselineskip,
    %frame=L,
    xleftmargin=\parindent,
    language=C++,
    basicstyle=\footnotesize\ttfamily,
    keywordstyle=\bfseries\color{blue},
    keywordstyle=[2]\bfseries\color{red},
    commentstyle=\itshape\color{mygreen},
    identifierstyle=\color{black},
    stringstyle=\color{gray},
    keywords=[2]{           % qt-Keywords
        Qt,
        SIGNAL,
        SLOT,
        QApplication,
        QDialog,
        QGridLayout,
        QPushButton,
        QLabel,
        QVBoxLayout,
        QHBoxLayout,
        QWidget,
        QGroupBox,
        QFont,
        QLineEdit,
        QRadioButton,
        QPen,
        QBrush,
        QPixmap,
        QPainter,
        QString},
}

\lstdefinestyle{cdoxy}{
    belowcaptionskip=1\baselineskip,
    %frame=L,
    xleftmargin=\parindent,
    language=C++,
    basicstyle=\footnotesize\ttfamily,   
    keywordstyle=\bfseries\color{blue},
    commentstyle=\itshape\color{mygreen},
    identifierstyle=\color{black},
    stringstyle=\color{gray},
    otherkeywords={           % DoxygenKeywords
        ...,
        ....,
        @mainpage,
        @file,
        @author,
        @version,
        @date,
        @bug,
        @brief,
        @extended,
        @param,
        @return,
        @warning,
        @note,
        @see},
}

\lstdefinestyle{customC}{
    belowcaptionskip=1\baselineskip,
    %frame=L,
    xleftmargin=\parindent,
    language=c,
    basicstyle=\footnotesize\ttfamily,   
    keywordstyle=\bfseries\color{blue},
    commentstyle=\itshape\color{mygreen},
    identifierstyle=\color{black},
    stringstyle=\color{gray},
    otherkeywords={           % c-Keywords
    size_t,
    memcpy},
}

\lstdefinestyle{customasm}{
    belowcaptionskip=1\baselineskip,
    %frame=L,
    xleftmargin=\parindent,
    language=[x86masm]Assembler,
    basicstyle=\footnotesize\ttfamily,
    commentstyle=\itshape\color{mygreen},
}

%choose customstyle in DOC with \lstinputlisting[style=custom]{path}
%\lstset{escapechar=@,style=customc++}
\lstset{style=customc++}
