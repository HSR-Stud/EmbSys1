\thispagestyle{empty}
\setcounter{page}{0} %Set PageNumber to 0
\vspace*{-2cm}
{\huge README }
{\scriptsize
\section*{Beschreibung}
Zusammenfassung für Embedded Systems 1 - Using Microcontrollers and the MSP430 auf Grundlage der Vorlesung HS 16 von Erwin Brändle  \newline
Bei Korrekturen oder Ergänzungen wendet euch an einen der Mitwirkenden.


\section*{Modulschlussprüfung}
Kompletter Stoff aus Skript, Vorlesung, Übungen und Praktikum

    \begin{itemize}
        \item Kapitel aus dem Buch Introduction to Embedded Systems 
	        \subitem 3.1-3.6 / 6.1-6.4 / 7 / 8.1-8.3 / 9 / 10 
	     \item Korrigenda zum Skript, falls eine solche vorliegt  
	     \item Übungen im Vorlesungsskript  
	     \item Inhalt aller Praktika (inkl. Pre-/Post-Lab Übungen)
	     \item Aufgaben im Stil der Übungen, Praktika und der im Unterricht gelösten Kurzaufgaben
    \end{itemize}

\textbf{Die Prüfung besteht aus 2 Teilen:}\newline
% \usepackage{array} is required
\begin{tabular}{p{1.5cm} p{3cm} p{10cm}}
    \textbf{ 1.Teil}   & closed Book & Theoretische Fragen zum ganzen Prüfungsinhalt \\ 
    \textbf{ 2.Teil}   & semi-open book & Aufgaben im Stil der Übungen, Praktika und der in den Vorlesungen gelösten Aufgaben \\ 
\end{tabular} \newline

Im 2.Teil erlaubte Hilfsmittel sind:
\begin{itemize}
	\item \textbf{max. 3 einseitig handgeschriebene A4 Seiten}
	\item Buch Introduction to Embedded Systems
    \subitem \url{http://link.springer.com/book/10.1007/978-1-4614-3143-5}
	\item Taschenrechner
\end{itemize}

Nicht erlaubt sind:
\begin{itemize}
	\item Skript
	\item Alte Prüfungen
	\item Übungen und Praktikumsunterlagen
\end{itemize}

\subsection*{Plan und Lerninhalte}
Fokus: MSP430 (Ultra Low Power Microcontroller) 
 
    \begin{itemize}
    	\item Power und Clock-System
    	\item Reset- und Bootstrap-Sequenzen  
    	\item Interrupt-Handling, Low-Power Modes, Speicher, DMA
    	\item GPIO, Capture/Compare, PWM, ADC, Comparator  
    	\item Watchdog und Timer
    	\item UART, SPI, I2C, usw.
    \end{itemize}

\vfill
\section*{Contributors}
\begin{tabular}{ll}
    Luca Mazzoleni& luca.mazzoleni@hsr.ch \\ 
    Michel Gisler & michel.gisler@hsr.ch \\ 
\end{tabular} 

    \section*{License}
    \textbf{Creative Commons BY-NC-SA 3.0}
    
    Sie dürfen:
    \begin{itemize}
        \item Das Werk bzw. den Inhalt vervielfältigen, verbreiten und öffentlich
        zugänglich machen.
        \item Abwandlungen und Bearbeitungen des Werkes bzw. Inhaltes anfertigen.
    \end{itemize}
    Zu den folgenden Bedingungen:
    \begin{itemize}
        \item Namensnennung: Sie müssen den Namen des Autors/Rechteinhabers in der von ihm
        festgelegten Weise nennen.
        \item Keine kommerzielle Nutzung: Dieses Werk bzw. dieser Inhalt darf nicht für
        kommerzielle Zwecke verwendet werden.
        \item  Weitergabe unter gleichen Bedingungen: Wenn Sie das lizenzierte Werk bzw. den
        lizenzierten Inhalt bearbeiten oder in anderer Weise erkennbar als Grundlage
        für eigenes Schaffen verwenden, dürfen Sie die daraufhin neu entstandenen
        Werke bzw. Inhalte nur unter Verwendung von Lizenzbedingungen weitergeben,
        die mit denen dieses Lizenzvertrages identisch oder vergleichbar sind.
    \end{itemize}
    Weitere Details: http://creativecommons.org/licenses/by-nc-sa/3.0/ch/
}
%If we meet some day, 
%and you think this stuff is worth it, you can buy me a beer in return.
\clearpage
\pagenumbering{arabic}% Arabic page numbers (and reset to 1)