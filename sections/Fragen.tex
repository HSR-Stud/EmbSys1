\section{Fragen}
\begin{itemize}
	\item \textbf{Unterschiede \& Gemeinsamkeiten zwischen Exception und Interrupt?}
	
	\item \textbf{Welche Info steht an der Adresse 0x0FFFC des MSP430?}
	
	\item \textbf{Was versteht man unter dem Begriff Auto-Vectored-Interrupt?}
	
	\item \textbf{Wie wird der Low Power Mode beim MSP430 aktiviert}
	
	\item \textbf{Wodurch kann die CPU aus dem LPM in den AktivMode gebracht werden? temporär, dauerhaft?}
	
	\item \textbf{Begriffe Erklären der seriellen Kommunikation?}
	\subitem Simplex
	\subitem Half-Duplex
	\subitem Full Duplex
	\subitem Single Ended
	\subitem Differntial
	
	\item \textbf{Was versteht man unter einem Null-Modem?}
	\subitem RX,TX gekreuzt um 2 Endgeräte zu verbinden. Hand-Shake Leitung
	
	\item \textbf{Unterschied RS-232 und UART?}
	\subitem RS-232 ist UART-Schnittstelle mit definiertem Pegel
	\subitem UART nicht definierte Logik-Pegel
	
	\item \textbf{Funktion und Notwendigkeit eines Ringbuffers?}
	\subitem Um Informationen zu buffern
	
	\item \textbf{Erläutern sie das XON/XOFF Handshake-Verfahren?}
	\subitem Wenn Empfangsbuffer voll wird sendet er ein Request an den Sender mit Stopp Senden, damit der Buffer nicht überläuft
	
	\item \textbf{Welche Signalleitungen sind für eine unidirektionale Bus-Verbindung von SPI-Master zu einem SPI-Slave minimal erforderlich}
	\subitem MOSI, SS, Clock
	
	\item \textbf{Wie kann ein Busteilnehmer auf dem I2C Bus die Geschwindigkeit der Datenübertragung reduzieren?}
	\subitem Die Clockleitung länger unten lassen
	
	\item \textbf{Aufgrund welcher Information erfolgt beim I2C die Bus-Arbitrierung (Buszuteilung)?}
	\subitem 
	
	\item \textbf{Wie erfolgt beim SPI-Bus die Bus-Arbitrierung?}
	\subitem 
	
	\item \textbf{Wie erfolgt bei einer UART die Bus-Arbitrierung?}
	\subitem
\end{itemize}




















